
The University of Texas (UTA) is the second largest university in the UT system with around 50,000 students. It is a comprehensive doctoral university located in the Dallas-Ft. Worth metroplex. HEP was selected as one of the first ''Organized Research Center of Excellence'' at UTA in 2011. PI De is the Director of the ORCE:HEP Center, which also includes faculty from commology, astrophysics, space sciences, and computtational sciences. The combined synergy of these activities, along with substantial commitment of university resources, provides strong support to the core DoE HEP mission at UTA. Overall, the university has invested over two million dollars to support HEP research activities.

A prime example of UTA's investment in science was the provision of the 120,000 sq.ft. Physics and Chemistry Research Building in 2006.  This building houses a high bay area for HEP, our ATLAS Tier 2 center, three detector development laboratories, an HEP conference room, faculty offices, and postdoc and graduate student offices. The building houses an excellent Physics mechanical workshop with the capabilities to manage large scale detector construction..

One finished lab space at UT Arlington's Physics and Chemistry Research Builiding is a 700 sq. feet lab space with the necessary ventilation for cryogenic experiments to take place. This lab space has recently been completed with a purification and pressure based gas recirculation system for liquid argon detector R$\&$D. This finished lab space also houses desk space, computers, soldering stations, and work space for the undergraduate detector sensor lab as well as a intensity frontier remote operations station. This remote operation station has already been used to take shifts on the LArIAT experiment and is being expanded for remote shift taking on MicroBooNE, SBND, and ICARUS. 

A 700 sq foot unfinished lab space adjacent to the purified liquid argon lab and located off the high-bay area has a 3 ton crane for detector construction and assembly. This lab space is located directly adjacent to the UTA physics department machine shop which can be used during detector testing and construction.

In addition to this lab space, the UTA HEP group have retained our previous office suite in Science Hall, and this area has been rennovated as the ATLAS Tier 2 operations and visitors area. The lab space in the basement of Science Hall now houses the purified gaseous xenon system as well as a laboratory space for testing Single Molecule Fluorescence Imaging.

UTA also hosts the SouthWest Tier 2 center (SWT2) for ATLAS, which is one of the largest computing centers for ATLAS, providing over 3000 cores and 3 petabytes of storage. The UTA HEP Tier 3 center is co-located with the Tier 2, providing easy access to ATLAS data.
