The UTA group currently is playing a leadership role across the Fermilab short-baseline (SBN) and long-baseline (LBN) neutrino programs. The UTA group, as of August 2018, consists of two postdocs, four graduate students, and two PI's funded from a mixture of base-grant, start-up funds, and teaching assistantships. This has allowed UTA to play successful roles across many of the liquid argon testbeam and neutrino experiments. Table \ref{tab:People} provides a high-level overview of the people and tasks currently being undertaken.


%%%%%%%%%%%%%%%%%%%%%%%%%%%%%%%%%%%%%%%%%%%%%%%%%%%%%%%
\begin{table}[htb]
    \begin{center}
    \resizebox{\textwidth}{!}{%
    \begin{tabular}{c|c|c|c}
    \multicolumn{4}{c}{\textbf{Summary of PI, Postdoc, and Graduate Personal}} \\
    \hline \hline
     \textbf{Personnel} & \textbf{Associated Task} & \textbf{Years Supported} & \textbf{Source of Support}   \\
    \hline
    %%%% PostDoc #1 %%%%%
    Postdoc 1  & protoDUNE DP Construction  & 2017 - 2020 & UTA Base Grant \\
    (Animesh Chatterjee) & ICARUS Commissioning and Operations & & \\
     & DUNE BSM Analysis & & \\
    \hline
    
    %%%% PostDoc #2 %%%%%
    Postdoc${*}$  & ICARUS Installation, Simulation, and Commissioning  & 2017 - 2019 & UTA Start-up funds \\
    (Andrea Falcone) & LArIAT Run Coordinator  & & \\
     & SBND CRT Construction & & \\
    \hline
    %%% Grad Student 1 %%%
    Graduate Student 1 $^{*}$ & SBND Cold Electronics Testing, & 2017 - 2018 & UTA Start-up funds \\
    (Zack Williams) & protoDUNE Cold Electronics Testing  & 2018 - 2020 & UTA Base Grant \\
     & MicroBooNE Operations and Data Analysis & & \\
    \hline
    %%% Grad Student 2 %%%
    Graduate Student 2a & MiniBooNE Data Analysis & 2017 - 2018 & UTA Base Grant \\
    (Sepideh Shahsavarani) & protoDUNE DP Construction & & \\
    \hline
    Graduate Student 2b & protoDUNE DP Commissioning & 2017 - 2018 & Teaching Assistantship \\
    (Hector Carranza) & ICARUS Commissioning and Operations & 2019 - 2020 & UTA Base Grant \\
     & protoDUNE/ICARUS Data Analysis & & \\
    \hline
    
    
    Prof. Jonathan Asaadi & SBN/DUNE & 2017 - 2020 & UTA Base Grant \\
     \hline
    Prof. Jae Yu & DUNE/SBN & 2017 - 2020 & UTA Base Grant \\
     \hline
    \end{tabular}}
    \caption{Table summarizing the personnel working on the various projects as described in the original proposal. Note: Personnel marked with ``*'' denote that their effort is supported for some phase of the project utilizing Asaadi's start-up funds. This is done to maximally leverage the UTA group across both DUNE and the SBN program.} \label{tab:People}
    \end{center}
\end{table}
%%%%%%%%%%%%%%%%%%%%%%%%%%%%%%%%%%%%%%%%%%%%%%%%%%%%%%%

As of September 2018, PI Asaadi's startup has been completely utilized and the funding support for UTA's second postdoc has concluded. In order to continue supporting the institutional responsibilities that UTA has on the ICARUS and SBND experiments, supplemental funding to the base grant is requested to support a postdoc through 2020. The efforts supported by this supplemental proposal will supply critical person-power needed by the ICARUS and SBND experiments as both enter the time critical period of installation, commissioning, and data taking. 


%%%%%%%%%%%%%%%%%%%%%%%%%%%%%%%%%%%%%%%%%%%%
\section*{Current Scope of Work}
%%%%%%%%%%%%%%%%%%%%%%%%%%%%%%%%%%%%%%%%%%%%

Tables \ref{tab:IFProjects} summarizes the projects and associated PI's taking a lead role during the current scope of work across the entire intensity frontier effort. In specific, the detector construction and installation roles on ICARUS and SBND are targeted for the requested supplemental proposal. The postdoctoral researcher to be hired will play a critical role on both the ICARUS and SBND experiments. These are roles that have been supported thus far via the startup funding for Andrea Falcone and will continue to be institutional responsibility for UTA over the next years.

%%%%%%%%%%%%%%%%%%%%%%%%%%%%%%%%%%%%%%%%%%%%%%%%%%%%%%
\begin{center}
\begin{table}[htb]
    \begin{center}
    \resizebox{0.85\textwidth}{!}{%
    \begin{tabular}{c|c|c|c}
    \multicolumn{4}{c}{\textbf{IF Summary Current Scope of Work}} \\
    \hline \hline
    \textbf{Experiment} & \textbf{Project} & \textbf{Location} & \textbf{Lead PI} \\
    \hline
         & protoDUNE SP-APA QA/QC and installation & UTA/CERN & Asaadi \\
    DUNE & BSM Physics & UTA & Yu \\
         & protoDUNE DP-FC Construction & UTA/CERN & Yu \\
    \hline
         & Cold Electronics TPC Test-stand & UTA/FNAL & Asaadi \\
    SBND & Detector Construction, Installation, and Commissioning & FNAL & Yu \\        
         & Cross-Section Data Analysis & UTA/FNAL & Asaadi \\
        
    \hline
    MicroBooNE & TPC Detector Expert & UTA/FNAL & Asaadi \\
               & Detector Operations & UTA/FNAL & Asaadi \\
               & Cross-Section Data Analysis & UTA/FNAL & Asaadi \\
    \hline
    ICARUS     & Detector Installation and Commissioning & FNAL & Asaadi \\
               & NuMI Off-Axis Cross-Sections $\&$ Low-mass Dark Matter & UTA/FNAL & Yu \\
              
    \hline

    MiniBooNE & Beam Dump Dark Matter Search & UTA/FNAL & Yu \\
    \hline                     
    \end{tabular}}
    \caption{Overview of the UTA projects across the Intensity Frontier} \label{tab:IFProjects}
    \end{center}
\end{table}
\end{center}
%%%%%%%%%%%%%%%%%%%%%%%%%%%%%%%%%%%%%%%%%%%%%%%%%%%%%%%

%%%%%%%%%%%%%%%%%%%%%%%%%%%%%%%%%%%%%%%%%%%%%%%%
\subsection*{Current ICARUS/SBND Accomplishments}
%%%%%%%%%%%%%%%%%%%%%%%%%%%%%%%%%%%%%%%%%%%%%%%%
The UTA postdoc currently supported via startup funds (Andrea Falcone) and under the mentorship of PI Asaadi and Yu has played a leadership role on the LArIAT and ICARUS experiments as well as providing supporting roles for construction of the Cosmic Ray Tagger (CRT) to be used in SBND. 

During Falcone's time at UTA (2015 - 2018) he took part in the upgrade of the ICARUS light detection system at CERN playing a central role in the testing of the PMT's and their installation into the cryostat \cite{Babicz:2018svg}, \cite{Bonesini:2018ubd}. UTA has also been involved in the simulation and testing of the readout electronics for the light detection system \cite{Bagby:2018fkj} as well as being one of the main drivers of the simulation of the scintillation light coming from neutrino interactions and PMT response within the full ICARUS Monte Carlo \cite{JPhys}. These accomplishments have positioned the UTA group to continue our role in the commissioning of the ICARUS detector and the expertise to continue to drive the simulation and readout of the ICARUS PMT and trigger system.

UTA has leveraged its leadership on the LArIAT detector to aid in the installation and data taking of the SBND Vertical Slice Test which saw the installation of prototype SBND cold front end motherboards on the LArIAT TPC. PI Asaadi was supported via a Fermilab Neutrino Physics Center Fellowship and worked in tandem with Falcone to recently help complete the first data taking period with the new electronics. Falcone has also spent a portion of his time working on the SBND CRT, making trips to the University of Bern to aid in the construction of the CRT panels. The experience working with the SBND readout electronics and CRT and UTA's considerable background working with the LArIAT experiment has positioned the group well to continue to make substantial contributions to the experiment working on the readout and triggering of the SBND experiment.

The UTA postdocs have had successful and visible leadership roles during their time at UTA and the researcher funded by this supplement will enter a group with a wealth of experience on both the ICARUS and SBND experiment allowing them to take on critical and timely tasks for both experiments.

  
%%%%%%%%%%%%%%%%%%%%%%%%%%%%%%%%%%%%%
\section*{Proposed Scope of Work}
%%%%%%%%%%%%%%%%%%%%%%%%%%%%%%%%%%%%%
The proposed scope of work to be carried out by the postdoc supported from this supplemental request is one already defined by the original proposal (DE-SC001168) by taking part in the construction, commissioning, and operation of the ICARUS and SBND experiments. The term for this appointment would be between November 2018 and March 2020 with the possibility of extension pending the outcome of UTA's the next competitive review. The postdoc is foreseen to be in residence at Fermilab throughout the duration of the position and thus will be able to give person-power to installation tasks as well as be among readout and simulation experts. The postdoc will have two UTA graduate students at Fermilab during this appointment, providing them both with a team of support and the opportunity to mentor and work with students.

Table \ref{tab:Timeline} provides and overview of the key tasks and timelines that are foreseen for the postdoc supported by this supplemental request. These times are divided along the current SBN schedule with installation and development tasks taking place between November 2018 and May 2019 and operations and deployment  foreseen between June 2019 and March 2020. The roles and responsibilities for these tasks are described in greater detail below.

%%%%%%%%%%%%%%%%%%%%%%%%%%%%%%%%%%%%%%%%%%%%%%%%%%%%%%
\begin{center}
\begin{table}[htb]
    \begin{center}
    \resizebox{0.85\textwidth}{!}{%
    \begin{tabular}{c|c|c|c}
    \multicolumn{4}{c}{\textbf{IF Summary of Proposed Work}} \\
    \hline \hline
    \textbf{Experiment} & \textbf{Project} & \textbf{Timeline} & \textbf{Time Commitment (FTE)} \\
    \hline
    ICARUS & TPC/PMT Electronics  & Nov 2018 - May 2019 & 0.4 \\
     & Installation and Testing & & \\
    \hline 
    SBND  & Trigger and Readout  & Nov 2018 - May 2019 & 0.4 \\ 
     & Testing and Development & & \\
    \hline
    ICARUS & Light and Charge Monte Carlo & Nov 2018 - May 2019 & 0.2 \\
     & Simulation and Trigger Studies & & \\
    \hline \hline
    SBND & Trigger and Readout & June 2019 - March 2020 & 0.4 \\
     & Deployment and Integration & & \\
    \hline
    ICARUS & Commissioning and Data Taking & June 2019 - March 2020 & 0.3 \\
     & Operations & & \\
    \hline
    ICARUS \& SBND & Integrated Physics & June 2019 - March 2020 & 0.3 \\
     & Software Trigger Development & & \\
     \hline            
    \end{tabular}}
    \caption{Overview of the UTA projects across the Intensity Frontier} \label{tab:Timeline}
    \end{center}
\end{table}
\end{center}
%%%%%%%%%%%%%%%%%%%%%%%%%%%%%%%%%%%%%%%%%%%%%%%%%%%%%%%


%%%%%%%%%%%%%%%%%%%%%%
\subsection*{Role on ICARUS}
%%%%%%%%%%%%%%%%%%%%%%
The UTA group has been playing a key role on the ICARUS experiment, as described above, involved in the light detection system and the simulation efforts. In addition to these roles, PI Asaadi has taken on the role of deputy TPC electronics convener (along side Sandro Centro from Padova, Matt Worcester from Brookhaven, and Mike Mooney from Colorado State University). 

The postdoc supported by this supplemental proposal will be leveraging this expertise to contribute to the installation and testing of the PMT and TPC electronics as they arrive at Fermilab and are installed on the cryostat. During these initial months both PI Asaadi and Yu expect to be spending extended periods at Fermilab to aid in the information transfer and to contribute directly to the ongoing installation work.

In addition to this, continued work on the simulation of the PMT and TPC signals within the ICARUS Monte Carlo is expected to continue. This simulation work plays a central role in the interest of developing a software based trigger to help distinguish localized physics signatures from the cosmic ray backgrounds through utilizing the PMT, TPC, and CRT information in an near-line software based trigger. This work is seen an synergistic with the trigger work on SBND and ultimately a unifying piece of the overall SBN physics program.

%%%%%%%%%%%%%%%%%%%%%%
\subsection*{Role on SBND}
%%%%%%%%%%%%%%%%%%%%%%
The UTA group has been contributing to the testing of various detector components for SBND, as described above, including ASIC chip testing, CRT assembly, and readout electronics testing with the SBND vertical slice test at LArIAT. Most recently UTA has partnered with the University of Pennsylvania group to play a leading role in the testing, development, and implementation of the Photon Trigger Board (PTB) in the SBND readout. This critical role allows UTA to leverage its experience with the various pieces of SBND hardware as well as our experience with the PMT system in ICARUS (which shares the same CAEN digitization boards and similar readout software).

The postdoc supported by this supplemental proposal is foreseen to play a leading role in the testing and implementation of the PTB beginning using the data acquisition test stands currently built at Fermilab. These test stands include the SBND TPC readout electronics, CAEN PMT digitizers (common between ICARUS and SBND), CRT frontend boards, and the Photon Trigger Board. This test stand will allow for the integration and testing of the various triggering schemes foreseen for SBND utilizing the ICARUS/SBND common artDAQ software framework  as well as exploration of an integrated software based trigger similar to that being used in ICARUS. 

In addition to playing a leadership role on the triggering and readout for the SBND experiment, the postdocs residence at Fermilab will allow them to participate in additional installation tasks as deemed necessary by the collaboration and as their other responsibilities allow. 

%%%%%%%%%%%%%%%%%
\section*{Conclusion}
%%%%%%%%%%%%%%%%%

A supplemental request to support a postdoctoral researcher to work on the SBND and ICARUS experiments from November 2019 through March 2020 will allow the UTA group to continue its leadership role on these experiments. The UTA group has leveraged PI Asaadi's startup funds to successfully execute their current scope of work and contribute the to Fermilab neutrino program. 

The person funded by this supplemental proposal will execute critical and well defined tasks on both the ICARUS and SBND experiment during the proposed period of work. In addition to providing needed person-power to the experiments during their installation, commissioning and operation phases, critical roles in the readout, triggering and simulations for the ICARUS and SBND experiments would go understaffed in the absence of this continued effort adversely effecting the experiments. 